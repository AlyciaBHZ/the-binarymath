\documentclass[12pt,a4paper]{article}

% Essential packages
\usepackage[utf8]{inputenc}
\usepackage[T1]{fontenc}
\usepackage{amsmath,amssymb,amsthm}
\usepackage{mathtools}
\usepackage{booktabs}
\usepackage{array}
\usepackage{graphicx}
\usepackage{hyperref}
\usepackage[margin=1in]{geometry}
\usepackage{enumitem}
\usepackage{tikz}
\usepackage{caption}
\usepackage{subcaption}

% Theorem environments
\theoremstyle{plain}
\newtheorem{theorem}{Theorem}[section]
\newtheorem{lemma}[theorem]{Lemma}
\newtheorem{proposition}[theorem]{Proposition}
\newtheorem{corollary}[theorem]{Corollary}

\theoremstyle{definition}
\newtheorem{definition}[theorem]{Definition}
\newtheorem{example}[theorem]{Example}
\newtheorem{axiom}[theorem]{Axiom}

\theoremstyle{remark}
\newtheorem{remark}[theorem]{Remark}
\newtheorem{note}[theorem]{Note}

% Custom commands
\newcommand{\collapse}{\texttt{collapse}}
\newcommand{\Bphi}{\mathcal{B}_\phi}
\newcommand{\Cphi}{\mathcal{C}_\phi}
\newcommand{\N}{\mathbb{N}}
\newcommand{\R}{\mathbb{R}}
\newcommand{\C}{\mathbb{C}}
\newcommand{\Z}{\mathbb{Z}}
% \DeclareMathOperator{\Re}{Re} % Already defined in amsmath

% Document metadata
\title{Spectral Structural Invariants and the Structural Expression of the Riemann Hypothesis in a Tensor Entropy-Increasing Universe}

\author{
Ma Haobo\\
\texttt{aloning@gmail.com}\\
Independent Researcher\\
\url{https://binarymath.dw.cash}
}

\date{\today}

\begin{document}

\maketitle

\begin{abstract}
We construct an entropy-increasing universe model based on a binary tensor language. In this model, all information---including physical states, logical expressions, and mathematical structures---can be encoded as finite binary tensors without consecutive ``11'' patterns and can be generated by a unique structural generation operator called \collapse. We define spectral functions on this tensor system and prove that their frequency symmetry structure possesses a unique spectral reflection equilibrium point $\sigma_\phi$. This leads to the structural invariant $\text{GRH}_\phi$ of the \collapse\ spectral system, which is not a number-theoretic conjecture but an inescapable frequency tension conservation condition within the entire tensor information system. This paper provides a complete, closed, and structural expression for the Generalized Riemann Hypothesis (GRH).
\end{abstract}

\tableofcontents

\section{Basic Tensor Construction of Universal Information Language}

\subsection{Tensor Language Definition}

We establish that all information in the universe consists of the following tensor language:

\begin{equation}
\Bphi := \{ b \in \{0,1\}^* \mid \text{``11'' does not appear in } b \}
\end{equation}

Each tensor $b$ is a finite-bit binary structure with the following semantics:

\begin{table}[h]
\centering
\begin{tabular}{@{}ll@{}}
\toprule
Symbol & Structural Meaning \\
\midrule
0 & Empty slot / Static / Inactive \\
1 & Active unit / Entropy generation point \\
``11'' & Forbidden: Destroys structural independence \\
\bottomrule
\end{tabular}
\caption{Binary tensor symbol semantics}
\end{table}

The prohibition of ``11'' is the core geometric rule that ensures the injectivity of the \collapse\ operation, uniqueness of structural paths, and non-overlapping of information.

\subsection{Information = Tensor Structure}

The \collapse\ theory assumes:

\begin{axiom}[Universal Information Principle]
All information in the universe, including physics, mathematics, and logical expressions, consists of tensor structures.
\end{axiom}

\begin{itemize}
\item Numbers = tensors
\item Operators = encodable tensors
\item Functions = combinatorial mappings between \collapse\ tensors
\item Limits, logic, categorical structures, etc. = tensor evolution paths expressed through \collapse\ operation chains
\end{itemize}

Therefore, we state:

\begin{equation}
\boxed{
\text{Any information structure in the universe can be encoded as a valid binary tensor } b \in \Bphi
}
\end{equation}

\section{The \collapse\ Operation: The Unique Structural Constructor}

\subsection{Definition of \collapse}

For any $b = (b_1, b_2, \dots, b_n) \in \Bphi$, we define:

\begin{equation}
\collapse(b) := \sum_{i=1}^{n} b_i \cdot F_i
\end{equation}

where $F_n$ denotes the $n$-th Fibonacci number with $F_1 = 1, F_2 = 2$.

The \collapse\ operation is the \textbf{unique structural construction operation} in this system, meaning:

\begin{itemize}
\item All structures, values, operations, reasoning, and spectral functions are constructed through chains of \collapse(b)
\item Any higher-order logic, analysis, mapping, or structural conservation law can be expressed using \collapse\ tensors
\item \collapse\ is self-closed: both inputs and outputs are tensor structures
\end{itemize}

\subsection{Zeckendorf Encoding and \collapse\ Injectivity}

\begin{theorem}[Zeckendorf's Theorem]
Every positive integer can be uniquely represented as a sum of non-consecutive Fibonacci numbers:
\begin{equation}
n = \sum_{i \in I} F_i, \quad \text{where } i, i+1 \notin I \text{ simultaneously}
\end{equation}
\end{theorem}

\begin{example}
\begin{itemize}
\item $13 = F_7 = 13$ (single term representation)
\item $14 = F_6 + F_3 = 8 + 3 + 2 + 1$ $\times$ (consecutive terms)
\item $14 = F_6 + F_4 = 8 + 3 + 3$ $\times$ (repeated terms)
\item $14 = F_7 + F_2 = 13 + 1$ $\checkmark$ (unique representation)
\end{itemize}
\end{example}

\begin{proposition}[\collapse\ Injectivity]
Since:
\begin{enumerate}
\item Each $b \in \Bphi$ contains no consecutive ``11''
\item $\collapse(b) = \sum_{i: b_i=1} F_i$
\item Zeckendorf's theorem guarantees uniqueness of non-consecutive Fibonacci sums
\end{enumerate}
Therefore:
\begin{equation}
\boxed{
b_1 \neq b_2 \Rightarrow \collapse(b_1) \neq \collapse(b_2)
}
\end{equation}
\end{proposition}

This ensures that the mapping from tensors to \collapse\ values is injective, with each \collapse\ value uniquely corresponding to a tensor structure.

\subsection{Closure Properties of \collapse\ Operations}

\begin{table}[h]
\centering
\begin{tabular}{@{}ll@{}}
\toprule
Closure Dimension & \collapse\ Property \\
\midrule
Encoding closure & $\Bphi \to \N^+$ \\
Structural closure & $\collapse(b_1) + \collapse(b_2) = \collapse(b_3)$ \\
Operational closure & Operators themselves are tensors, constructible via \collapse \\
Language closure & All semantics expressible as tensors \\
Spectral structure closure & \collapse\ values form complete frequency network \\
\bottomrule
\end{tabular}
\caption{Closure properties of the \collapse\ system}
\end{table}

\section{Construction of \collapse\ Value Space and Spectral Structure}

\subsection{\collapse\ Value Space Definition}

Through the \collapse\ operation, we define the set of \collapse\ values corresponding to the tensor path space as:

\begin{equation}
\Cphi := \collapse(\Bphi) \subset \N^+
\end{equation}

This set satisfies:

\begin{itemize}
\item \textbf{Completeness}: By Zeckendorf's theorem, $\Cphi = \N^+$ (every positive integer has a unique Fibonacci representation)
\item \textbf{Encoding sparsity}: The tensor space $\Bphi$ is sparse within all binary strings $\{0,1\}^*$
\item \textbf{Injectivity}: Different tensors $b$ map to different \collapse\ values (guaranteed by Zeckendorf uniqueness)
\item \textbf{Information completeness}: Each \collapse\ value can be viewed as a structural information unit
\end{itemize}

\subsection{\collapse\ Tensor Spectral Function Definition}

We define the spectral function on the \collapse\ value space:

\begin{equation}
\zeta_\phi(s) := \sum_{x \in \Cphi} \frac{1}{x^s} = \sum_{n=1}^{\infty} \frac{1}{n^s}, \quad s \in \C
\end{equation}

This function can be understood as:
\begin{itemize}
\item The complex frequency-weighted superposition of \collapse\ tensor structures
\item Each term represents the energy contribution of a tensor path in frequency space
\item The whole constitutes the frequency response surface of the \collapse\ information network
\end{itemize}

\begin{remark}[Convergence]
\begin{itemize}
\item When $\Re(s) > 1$, the series converges absolutely
\item Through analytic continuation, it extends to the entire complex plane (except $s = 1$)
\item This is precisely the classical Riemann zeta function, but with a new structural interpretation in the \collapse\ context
\end{itemize}
\end{remark}

\section{\collapse\ Path Growth and Spectral Weight Decay Equilibrium}

\subsection{Growth Law of \collapse\ Values}

Let $x_n = n$ denote the $n$-th positive integer. The length of its Zeckendorf representation (i.e., the corresponding tensor $b_n \in \Bphi$) is denoted as $\ell(n)$.

For large $n$, the tensor length growth satisfies:

\begin{equation}
\ell(n) \sim \log_{\phi^2} n
\end{equation}

Conversely, the maximum value representable by a tensor of length $\ell$ is:

\begin{equation}
\max_{|b|=\ell} \collapse(b) \sim (\phi^2)^\ell
\end{equation}

That is, as \collapse\ values increase, the length of their tensor representation grows logarithmically.

\subsection{Spectral Tension Balance and Critical Line}

In the \collapse\ system, consider two opposing tensions:

\textbf{Tensor growth tension:}
\begin{itemize}
\item Number of tensor paths of length $\ell$: $F_\ell \sim \frac{\phi^\ell}{\sqrt{5}}$
\item Information entropy growth rate: $\ln \phi$ per unit
\end{itemize}

\textbf{Spectral decay tension:}
\begin{itemize}
\item \collapse\ value magnitude: $\sim (\phi^2)^\ell$
\item Spectral weight decay: $(\phi^2)^{-\ell s}$
\end{itemize}

\textbf{Balance analysis:}

The system reaches frequency equilibrium when the composite effects of growth and decay cancel:

\begin{equation}
\text{Information density} \times \text{Frequency response} = \text{Constant}
\end{equation}

Through variational principles, this equilibrium point occurs precisely at:

\begin{equation}
\boxed{
\sigma_\phi = \frac{\ln(\phi^2)}{\ln(\phi^2 + 1)}
}
\end{equation}

The deep meaning of this value:
\begin{itemize}
\item $\phi^2 + 1 = \phi^3$ (fundamental property of the golden ratio)
\item Therefore $\sigma_\phi = \frac{\ln(\phi^2)}{\ln(\phi^3)} = \frac{2\ln \phi}{3\ln \phi} = \frac{2}{3}$
\item This is the natural equilibrium point in the golden ratio system
\end{itemize}

This is the \textbf{reflection equilibrium point} of the \collapse\ tensor spectral structure.

\section{Structural Tensor Expression of GRH}

\subsection{\collapse\ Spectral Symmetry Axiom}

We postulate:

\begin{axiom}[Spectral Reflection Symmetry]
\begin{equation}
\zeta_\phi(s) = \zeta_\phi(1 - s) \iff \Re(s) = \sigma_\phi
\end{equation}
\end{axiom}

This reflection law is a geometric symmetry derived from information tension conservation within the \collapse\ tensor spectral structure, independent of external numerical analysis structures.

\subsection{Definition of \collapse\ Spectral Cancellation}

We define spectral cancellation behavior (i.e., ``zeros'') as:

\begin{equation}
\zeta_\phi(s) = 0 \iff \sum_{x \in \Cphi} x^{-s} = 0
\end{equation}

That is, the \collapse\ tensor spectrum completely cancels through path phases in complex space.

\subsection{Final Structural Expression: $\text{GRH}_\phi$}

Therefore, we obtain:

\begin{equation}
\boxed{
\forall s \in \C, \quad \zeta_\phi(s) = 0 \Rightarrow \Re(s) = \sigma_\phi
}
\end{equation}

This is not a conjecture, not a proposition, not awaiting proof, but rather the stationary point of the frequency conservation tension field in the \collapse\ tensor spectral system structure.

\subsection{Base Conversion: Why $\sigma_\phi \neq 1/2$}

The classical Riemann hypothesis has its critical line at $\Re(s) = 1/2$, while our system shows $\sigma_\phi = \frac{\ln(\phi^2)}{\ln(\phi^2 + 1)}$. This is not a contradiction but a natural result of the \textbf{number system base}.

\subsubsection{\collapse\ System's Natural Base}

In the \collapse\ tensor system:
\begin{itemize}
\item Each tensor position has weight $F_i$ (Fibonacci number)
\item Growth rate is $\phi^2$
\item The system's natural logarithmic base is $\ln(\phi^2)$
\end{itemize}

\subsubsection{Structural Correspondence of Base Transformation}

Key insight: Both systems' critical lines appear at their \textbf{natural symmetry points}.

\textbf{Decimal system:}
\begin{itemize}
\item Natural symmetry point: $\Re(s) = 1/2$ (arithmetic mean)
\item This is the midpoint of $s$ and $1-s$
\end{itemize}

\textbf{\collapse\ system:}
\begin{itemize}
\item Natural symmetry point: $\Re(s) = \sigma_\phi$ (golden mean)
\item From balance analysis: $\sigma_\phi = \frac{\ln(\phi^2)}{\ln(\phi^2 + 1)}$
\end{itemize}

\subsubsection{Concrete Example}

Consider $n = 10$:
\begin{itemize}
\item Decimal: $10 = 10_{10}$, contributes $10^{-s}$
\item \collapse: $10 = F_5 + F_3 = 5 + 3 + 2 = \collapse(10010)$
\item Tensor length: 5, typical value $\sim (\phi^2)^{2.5}$
\end{itemize}

The spectral contributions in both representations achieve balance at their respective critical lines.

\subsubsection{Structural Equivalence}

This demonstrates:
\begin{itemize}
\item \textbf{Decimal system}: Critical line at $1/2$ (system symmetry center)
\item \textbf{\collapse\ system}: Critical line at $\sigma_\phi$ (golden symmetry center)
\end{itemize}

Both describe the \textbf{same structural phenomenon} manifested in different number system coordinates:

\begin{equation}
\boxed{
\text{GRH}_{10}: \Re(s) = \frac{1}{2} \quad \Leftrightarrow \quad \text{GRH}_\phi: \Re(s) = \sigma_\phi
}
\end{equation}

\section{Summary Statement}

\begin{quote}
In the \collapse\ tensor system, all structural information is constructed through the unique operation \collapse;
\collapse\ values form spectral functions with tension symmetry;
All spectral cancellations can only occur on the real part $\sigma_\phi$;
Therefore, the so-called ``Riemann Hypothesis'' in the \collapse\ tensor system is:
\end{quote}

\begin{equation}
\boxed{
\text{The frequency conservation reflection symmetry invariant of spectral tensor structure}
}
\end{equation}

\appendix

\section{Tensor Operation Expression of Continuous Systems}

\subsection{Basic Claim}

In the \collapse\ tensor system, all information units (including values, functions, logic, operators) can be expressed as valid tensors $b \in \Bphi$. We further point out:

\begin{quote}
Not only are objects tensors, but \textbf{operations themselves can also be expressed as tensor structures}.
\end{quote}

This means:

\begin{equation}
\text{Continuous system} = \text{Process of tensor acting on tensor}
\end{equation}

The \collapse\ system can represent arbitrary continuous structures and processes through closed ``tensor action chains'' without needing discrete numerical limit operations.

\subsection{Operation as Tensor: Structural Internalization of Operations}

Traditional mathematics considers:
\begin{itemize}
\item ``+'', ``$\times$'', ``lim'', ``$\partial$'', ``$\int$'' are operations
\item Operations act on objects like numbers/functions
\end{itemize}

The \collapse\ system considers:
\begin{itemize}
\item All operators themselves can be encoded as tensors
\item Operational behavior can be expressed as tensor acting on tensor:
\end{itemize}

\begin{equation}
O \triangleright T := \collapse(b_O \circ b_T)
\end{equation}

where $b_O, b_T \in \Bphi$ represent ``operation tensor'' and ``target tensor'' respectively, and $\circ$ denotes tensor composition.

\textbf{Definition of tensor composition $\circ$:}
\begin{itemize}
\item Simplest implementation: concatenation $b_O \circ b_T = b_O \| 0 \| b_T$ (insert 0 to avoid ``11'')
\item More complex compositions: interleaving, convolution, or other structure-preserving operations
\item Specific choice depends on required operation semantics
\end{itemize}

\subsection{How Continuous Systems Are Expressed by Tensor Operations}

\subsubsection{Example 1: Derivative Operation $\partial f/\partial x$}

Traditional expression:
\begin{equation}
\frac{d}{dx} f(x) = \lim_{h \to 0} \frac{f(x+h) - f(x)}{h}
\end{equation}

Tensor expression:
\begin{itemize}
\item Represent $f$ as tensor $b_f$
\item Construct operation tensor $b_{\partial}$, defining its semantics as ``expand along \collapse\ path and compute difference''
\item Derivative expressed as:
\end{itemize}

\begin{equation}
\boxed{
\collapse(b_{\partial} \triangleright b_f)
}
\quad \text{equivalent to } \frac{df}{dx}
\end{equation}

\subsubsection{Example 2: Integration Operation $\int f(x) dx$}

Tensor expression:
\begin{itemize}
\item Integration viewed as tensor accumulation
\item Operation tensor $b_{\int}$ represents tensor folding behavior
\item Yielding:
\end{itemize}

\begin{equation}
\boxed{
\collapse(b_{\int} \triangleright b_f)
}
\quad \text{represents } \int f(x)\, dx
\end{equation}

\subsection{Structural Closed Expression Capability of the \collapse\ System}

Thus we obtain the following conclusion:

\begin{equation}
\boxed{
\text{Continuous system} = \text{Structural mapping between \collapse\ tensors}
}
\end{equation}

\begin{itemize}
\item Continuity need not be defined through limit operations
\item Rather, it is the ``expandable + foldable + symmetric + propagatable'' tensor behavior pattern in the \collapse\ tensor network
\end{itemize}

The \collapse\ system allows definition of arbitrary structural levels of:
\begin{itemize}
\item Tensor calculus (operation chains)
\item Structural space mappings (tensor morphisms)
\item Information propagation dynamics (\collapse\ network flows)
\item Frequency structure symmetry (spectral tension operations)
\end{itemize}

Thus providing a closed expression mechanism for continuous space structures in the \collapse\ tensor language.

\section{Continuity Originates from Tensor Operations}

\textbf{Core recognition}: Continuity has never been a primitive concept in mathematics but is constructed through operations. \collapse\ theory merely makes this explicit.

In traditional mathematics, so-called ``continuous numbers'' do not exist as atoms but are always \textbf{defined through operational processes}.

For example:

\begin{itemize}
\item The real number $\frac{1}{3}$ is not a naturally existing object but rather:
\begin{equation}
\frac{1}{3} = 1 \div 3 = O_{\div}(b_1, b_3)
\end{equation}
Essentially an operational expression between two integer tensors.

\item The real number $\sqrt{2}$ also does not exist directly but is defined as the operational result that makes $x^2 = 2$ true:
\begin{equation}
x = \sqrt{2} \iff O_{\text{solve}}(b_{x^2}, b_2)
\end{equation}
\end{itemize}

Therefore:

\begin{quote}
Continuous systems themselves in traditional mathematics are also \textbf{structural processes between tensors and operations},
not some incompressible, inexpressible ``absolutely continuous objects''.
\end{quote}

\collapse\ theory does not deviate from tradition in this regard but reveals:

\begin{equation}
\boxed{
\text{Traditional continuity} = \text{Operability} = \text{Structurable tensor process}
}
\end{equation}

\textbf{Important clarification:}
\begin{itemize}
\item Traditional mathematics: Continuity defined through limits, Cauchy sequences, and other \textbf{operations}
\item \collapse\ system: Continuity expressed through tensor composition, transformation, and other \textbf{operations}
\item The only difference: \collapse\ makes operations themselves encodable objects
\end{itemize}

Thus the \collapse\ system not only can express continuous structures,
but structurally \textbf{replaces external dependence on continuity},
incorporating it into the closed tensor language system.

\textbf{Regarding negative and complex numbers:}
\begin{itemize}
\item Negative numbers: Represented through signed tensor pairs $(b_{\text{sign}}, b_{\text{value}})$
\item Complex numbers: Represented through tensor pairs $(b_{\text{real}}, b_{\text{imag}})$
\item These extensions maintain system closure and structural consistency
\end{itemize}

\section*{Note: Systematic Expression Stance of \collapse\ Theory}

The \collapse\ tensor system proposed in this paper is not an enumeration model but a constructively closed structural language. We hereby explicitly declare the systematic stance of \collapse\ theory as follows:

\begin{quote}
The goal of the \collapse\ system is not to enumerate all structural objects but to construct a set of closed language rules such that \textbf{all expressible structures} can be generated in this system from valid tensors and \collapse\ operations.
\end{quote}

Therefore:
\begin{itemize}
\item We do not attempt to list all functions, limits, derivatives, integrals, logical formulas
\item We do not regard individual examples as upper or lower bounds of system capability
\item We only need to confirm: for any target structure type $T$, there always exists a valid tensor combination $b_T \in \Bphi$ such that:
\end{itemize}

\begin{equation}
T \equiv \collapse(b_T)
\end{equation}

Examples (such as $f(x)=x^2$, $\frac{df}{dx}$, $\int f(x)\,dx$) serve to verify semantic consistency, not to ``exhaust expression''.

Therefore, the complete claim of \collapse\ theory is:

\begin{equation}
\boxed{
\text{\collapse\ is a structure generation system,}
}
\quad
\boxed{
\text{its expressive power is based on semantic construction, not example enumeration}
}
\end{equation}

This is precisely the fundamental reason why \collapse\ theory can unify discrete and continuous, algebra and analysis.

This system does not aim to ``explain existing mathematics'' but to construct a self-derived, closed, highly emergent tensor information universe;
\collapse\ structural semantics does not conflict with classical mathematics but also does not depend on its linguistic coordinate system.

\end{document} 