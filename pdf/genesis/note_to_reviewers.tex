
\textbf{Review Note}: This paper proposes a theoretical framework based on a single axiom---\textbf{self-referentially complete systems necessarily increase in entropy}. This axiom explicitly defines all core concepts, including self-referential completeness, entropy, information equivalence, etc. We employ an axiomatic method, starting from this clearly defined single axiom and exploring its theoretical consequences through logical derivation.

\textbf{Important Declaration}:

\begin{enumerate}
\item \textbf{Methodological Positioning}: This work adopts the same methodology as major discoveries in physics history:
\end{enumerate}
   - Newton derived the law of universal gravitation from the assumption that "all things attract each other"
   - Einstein derived relativity from the assumption of "constant speed of light"
   - We \textbf{construct} a single axiom with all fundamental concepts explicitly defined
   - From this clearly defined single axiom, we develop the theory through rigorous logical derivation

\begin{enumerate}
\item \textbf{Nature of the Theory}: This is a \textbf{constructive axiom-deductive} system. We find that from this single axiom, we can derive:
\end{enumerate}
   - Observer mechanisms consistent with quantum mechanics
   - Encoding systems ($\phi$-representation) consistent with information theory
   - Mathematical structures formally similar to the Riemann Hypothesis
   
   These similarities may suggest deep connections, but could also be mere mathematical coincidences.

\begin{enumerate}
\item \textbf{Constructive View of Truth}:
\end{enumerate}
   - We do \textbf{not} claim to have "discovered" the "real" structure of the universe
   - We \textbf{acknowledge} the theory is constructed, but not arbitrary
   - The theory's value lies in its \textbf{internal consistency} and \textbf{explanatory power}
   - The observer plays a fundamental role in theoretical construction

\begin{enumerate}
\item \textbf{Reading Guide}:
\end{enumerate}
   - Please view this paper as a \textbf{thought experiment}: What conclusions follow if we accept the basic axiom?
   - Focus on the \textbf{logical rigor of derivation}, not the absoluteness of conclusions
   - Appreciate the theoretical beauty from \textbf{minimal assumptions} to \textbf{maximal explanatory power}

\textbf{Core Point}: This paper demonstrates how to build an internally consistent theoretical framework from assumptions about self-referential systems. This framework produces results consistent with known physics and mathematics---a consistency that may merit further exploration.

