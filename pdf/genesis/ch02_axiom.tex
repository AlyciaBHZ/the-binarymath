
\textbf{Unique Axiom: Self-referentially complete systems necessarily increase in entropy}

\section{Complete Formal Statement of the Axiom}
\label{sec:ch02_axiom:complete-formal-statement-of-the-axiom}

\begin{tcolorbox}[colback=gray!10,colframe=black]
\begin{equation*}
\begin{aligned}
&\text{Unique Axiom: Self-referentially complete systems necessarily increase in entropy} \\
&\text{SelfRefComplete}(S) \Rightarrow \forall t \in \mathbb{N}: H(S_{t+1}) > H(S_t) \\
&\text{where the following definitions clarify the meaning of concepts in the axiom:}
\end{aligned}
\end{equation*}
\end{tcolorbox}

\section{Basic Structure Definitions (Clarification of Concepts in the Axiom)}
\label{sec:ch02_axiom:basic-structure-definitions-clarification-of-concepts-in-the-axiom}

\begin{itemize}
\item $\mathcal{S}$: Set of all possible states (containing objects, functions and their representations)
\item $S_t \subseteq \mathcal{S}$: Set of states contained in the system at time t
\item $\mathcal{L} \subseteq \mathcal{S}$: Formal language, i.e., set of finite symbol strings, a subset of state space
\item $t \in \mathbb{N}$: Discrete time parameter
\end{itemize}

\textbf{Ontological Clarification}: $\mathcal{S}$ contains four types of elements:
\begin{enumerate}
\item Basic objects (e.g., initial state $s_0$)
\item Representations of functions (e.g., encoding of $\text{Desc}$)
\item Description results (e.g., symbol strings produced by $\text{Desc}(s)$)
\item Symbol strings themselves (elements of formal language $\mathcal{L}$)
\end{enumerate}

\textbf{Key Relations}:
\begin{itemize}
\item $\mathcal{L} \subseteq \mathcal{S}$: Symbol strings are also possible states
\item $\text{Desc}: S_t \to \mathcal{L} \subseteq \mathcal{S}$: Results of description remain in state space
\item At any moment the system may contain certain symbol strings: $\mathcal{L} \cap S_t$ may be non-empty
\end{itemize}

\section{Definition of Self-Referential Completeness (Clarification of "SelfRefComplete" in the Axiom)}
\label{sec:ch02_axiom:definition-of-self-referential-completeness-clarification-of-selfrefcomplete-in-the-axiom}

\begin{equation}
\text{SelfRefComplete}(S) \equiv \exists \text{Desc}: S \to \mathcal{L} \text{ satisfying:}
\end{equation}

\begin{enumerate}
\item \textbf{Completeness}: $\forall s_1, s_2 \in S: s_1 \neq s_2 \Rightarrow \text{Desc}(s_1) \neq \text{Desc}(s_2)$
\end{enumerate}
   (The description function is injective on S)

\begin{enumerate}
\item \textbf{Containment}: $[\text{Desc}] \in S$ 
\end{enumerate}
   (The representation $[\text{Desc}]$ of the description function is part of the system)

\begin{enumerate}
\item \textbf{Self-reference}: $\exists d \in \mathcal{L}: d = \text{Desc}([\text{Desc}]) \land d \in \text{Range}(\text{Desc})$
\end{enumerate}
   (The description function can describe its own representation)

\begin{enumerate}
\item \textbf{Recursive Closure}: $\text{Desc}(s) \in \mathcal{L} \subseteq \mathcal{S}$ means the result of description is itself a possible system state,
\end{enumerate}
   thus $\text{Desc}(\text{Desc}(s))$ is a meaningful operation

\section{Definition of Entropy (Clarification of "H" in the Axiom)}
\label{sec:ch02_axiom:definition-of-entropy-clarification-of-h-in-the-axiom}

\begin{equation}
H(S_t) \equiv \log |\{d \in \mathcal{L}: \exists s \in S_t, d = \text{Desc}_t(s)\}|
\end{equation}

That is, the logarithm of the number of different descriptions in the system.

\section{Meaning of Entropy Increase (Clarification of "necessarily increase in entropy" in the Axiom)}
\label{sec:ch02_axiom:meaning-of-entropy-increase-clarification-of-necessarily-increase-in-entropy-in-the-axiom}

\begin{equation}
\text{Entropy Increase} \equiv \forall t \in \mathbb{N}: H(S_{t+1}) > H(S_t)
\end{equation}

\section{Information Equivalence Principle (Technical Clarification of the Axiom)}
\label{sec:ch02_axiom:information-equivalence-principle-technical-clarification-of-the-axiom}

In self-referential systems, states $s_1, s_2$ are informationally equivalent if and only if they are indistinguishable under the description function:

\begin{equation}
\text{InfoEquiv}(s_1, s_2) \equiv \text{Desc}(s_1) = \text{Desc}(s_2)
\end{equation}

This principle ensures:
\begin{itemize}
\item The injectivity of the description function applies to informationally different states
\item Physically identical states can have the same description
\item Avoids formal paradox problems
\end{itemize}

\textbf{Ontological Consistency}: Since $\mathcal{L} \subseteq \mathcal{S}$, the result of description $\text{Desc}(s) \in \mathcal{L}$ is itself a possible system state, which ensures:
\begin{itemize}
\item The system can contain descriptions of its own descriptions
\item The recursive operation $\text{Desc}(\text{Desc}(s))$ is ontologically meaningful
\item Self-referential completeness does not encounter type errors
\end{itemize}

\section{Philosophical Status of the Single Axiom}
\label{sec:ch02_axiom:philosophical-status-of-the-single-axiom}

\textbf{Constructive Declaration}:
\begin{itemize}
\item We \textbf{chose} this single axiom as the theoretical foundation
\item The definitions of entropy, self-referential completeness, etc. in the axiom are all \textbf{explicitly specified} by us
\item Key ontological choice: $\mathcal{L} \subseteq \mathcal{S}$ (symbol strings are also states)
\item The axiom's value lies in its \textbf{internal consistency} and \textbf{explanatory power}
\item We do not claim to have "discovered" the "real" structure of the universe, but rather \textbf{constructed} a self-consistent theoretical framework
\end{itemize}

\textbf{Role of the Observer}:
\begin{itemize}
\item The entire theory is constructed within the observer's cognitive framework
\item The observer chose the ontology unifying symbol strings and states
\item This choice makes self-referential completeness technically feasible
\end{itemize}

\section{System Evolution Mechanism (Clarification of Time Evolution in the Axiom)}
\label{sec:ch02_axiom:system-evolution-mechanism-clarification-of-time-evolution-in-the-axiom}

\textbf{Time Parameter}: $t \in \mathbb{N}$ is a discrete time step, naturally emerging from self-referential recursion

\textbf{State Evolution Rule}: $S_{t+1} = \Phi(S_t)$, where the evolution operator $\Phi$ is defined as:

\begin{equation}
\Phi(S_t) = S_t \cup \{\text{new description layer}\} \cup \{\text{recursively generated new states}\}
\end{equation}

Specifically, the new description layer includes:
\begin{itemize}
\item Description of $S_t$ as a whole: $\text{Desc}^{(t+1)}(S_t) \in \mathcal{L} \subseteq \mathcal{S}$
\item Descriptions of existing descriptions: ${\text{Desc}(d) : d \in S_t \cap \mathcal{L}}$
\item Recursive chains: Higher-order descriptions like $\text{Desc}(\text{Desc}(s))$
\end{itemize}

\textbf{Key Insight}: Since $\mathcal{L} \subseteq \mathcal{S}$, results of description can become inputs for the next round of description, forming a true recursive structure.

\textbf{Note}: $\text{Desc}_t$ denotes the description function at time t, which can evolve with the system.

\section{Five-fold Equivalent Formulation of the Single Axiom}
\label{sec:ch02_axiom:five-fold-equivalent-formulation-of-the-single-axiom}

In our theoretical framework, this single axiom is logically equivalent to the following formulations:

\begin{enumerate}
\item \textbf{Entropy formulation}: If a system can describe itself, its description diversity (by our definition) irreversibly increases
\item \textbf{Time formulation}: Self-referential structure necessarily leads to irreversible structure $\Rightarrow$ emergence of time
\item \textbf{Observer formulation}: If descriptor $\in$ system $\Rightarrow$ observation behavior necessarily affects system state
\item \textbf{Asymmetry formulation}: $S_t \neq S_{t+1}$, because each recursion adds irreducible information structure
\item \textbf{Structure formulation}: The system irreversibly unfolds along recursive paths
\end{enumerate}

These equivalences show that in our constructed theoretical framework, entropy increase, asymmetry, time, information, and observers can be understood as different aspects of the same phenomenon.

